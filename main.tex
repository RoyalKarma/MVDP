%%%%%%%%%%%%%%%%%%%%%%
% NASTAVENÍ FEKT.TEX %
%%%%%%%%%%%%%%%%%%%%%%

% Pokud následující řádky zakomentujete, na titulní straně se nezobrazí.

% Nadpis dokumentu (kód předmětu)
\newcommand{\name}{BPC-VDP}
% Podnadpis dokumentu (název předmětu)
%\newcommand{\subname}{}
% Seznam autorů
\newcommand{\authors}{Karma}
% Seznam korektorů
%\newcommand{\corrections}{}
% Popis dokumentu
\newcommand{\docdesc}{Otázky ke zkoušce}
% Zařazení dokumentu (studijní program)
\newcommand{\docgroup}{Informační bezpečnost, FEKT VUT}
% Odkaz
\newcommand{\docurl}{https://github.com/VUT-FEKT-IBE/FEKT.tex}

% Přepsáním argumentu na 'true' zapnete balíček 'minted' pro sázení kódu.
% Pro jeho použití lokálně musíte mít v systému dostupný Python 3, python
% knihovnu 'minted' a PDFLaTeX musíte spouštět s argumentem '-shell-escape'.
% Místo něj můžete použít prostředí 'lstlisting'.
\newcommand{\docminted}{false}


%%%%%%%%%%%%%%%%%%%%
% OBECNÉ NASTAVENÍ %
%%%%%%%%%%%%%%%%%%%%

\newcommand{\fekttexversion}{2.0}

\documentclass[
    % Velikost základního písma je 12 bodů
    12pt,
    % Formát papíru je A4
    a4paper,
    % Oboustranný tisk
    twoside,
    % Záložky a metainformace ve výsledném PDF budou v kódování unicode
    unicode,
]{article}

% Kódování zdrojových souborů
\usepackage[utf8]{inputenc}
% Kódování výstupního souboru
\usepackage[T1]{fontenc}
% Podpora češtiny
\usepackage[czech]{babel}

% Geometrie stránky
\usepackage[
    % Horní a dolní okraj
    tmargin=25mm,
    bmargin=25mm,
    % Vnitřní a vnější okraj
    lmargin=30mm,
    rmargin=20mm,
    % Velikost zápatí
    footskip=17mm,
    % Vypnutí záhlaví
    nohead,
]{geometry}

% Zajištění kopírovatelnosti a prohledávanosti vytvořených PDF
\usepackage{cmap}
% Podmínky (pro použití v titulní straně)
\usepackage{ifthen}

%%%%%%%%%%%%%%%
% FORMÁTOVÁNÍ %
%%%%%%%%%%%%%%%

% Nastavení stylu nadpisů
\usepackage{sectsty}
% Formátování obsahů
\usepackage{tocloft}
\setcounter{tocdepth}{1}
% Odstranění mezer mezi řádky v seznamech
\usepackage{enumitem}
\setlist{nosep}
\setitemize{leftmargin=1em}
\setenumerate{leftmargin=1.5em}
\renewcommand{\labelitemi}{--}
\renewcommand{\labelitemii}{$\circ$}
\renewcommand{\labelitemiii}{$\cdot$}
\renewcommand{\labelitemiv}{--}
% Sázení správných uvozovek pomocí '\enquote{}'
\usepackage{csquotes}
% Vynucení umístění poznámek pod čarou vespod stránky
\usepackage[bottom]{footmisc}
% Automatické zarovnání textu k předcházení vdov a parchantů
\usepackage[defaultlines=3,all=true]{nowidow}
% Zalomení části textu pokud není na současné stránce dost místa
\usepackage{needspace}
% Nastavení řádkování
\usepackage{setspace}
\onehalfspacing
% Změna odsazení odstavců
\setlength{\parskip}{1em}
\setlength{\parindent}{0em}

% Bezpatkové sázení nadpisů
\allsectionsfont{\sffamily}
% Změna formátování nadpisu a podnadpisů v Obsahu
\renewcommand{\cfttoctitlefont}{\Large\bfseries\sffamily}
\renewcommand{\cftsubsecdotsep}{\cftdotsep}

% Použití moderní/aktualizované sady písem
\usepackage{lmodern}

%%%%%%%%%%%
% NADPISY %
%%%%%%%%%%%

\usepackage{titlesec}

\titlespacing*{\section}{0pt}{10pt}{-0.2\baselineskip}
\titlespacing*{\subsection}{0pt}{0.2\baselineskip}{-0.2\baselineskip}
\titlespacing*{\subsubsection}{0pt}{0.2\baselineskip}{-0.2\baselineskip}
\titlespacing*{\paragraph}{0pt}{0pt}{1em}

%%%%%%%%%%
% ODKAZY %
%%%%%%%%%%

% Tvorba hypertextových odkazů
\usepackage[
    breaklinks=true,
    hypertexnames=false,
]{hyperref}
% Nastavení barvení odkazů
\hypersetup{
    colorlinks,
    citecolor=black,
    filecolor=black,
    linkcolor=black,
    urlcolor=blue
}

%%%%%%%%%%%%%%%%%%%%%%%%%%%
% OBRÁZKY, GRAFY, TABULKY %
%%%%%%%%%%%%%%%%%%%%%%%%%%%

% Vkládání obrázků
\usepackage{graphicx}
\usepackage{subfig}
% Nastavení popisů obrázků, výpisů a tabulek
\usepackage{caption}
\captionsetup{justification=centering}
% Grafy a vektorové obrázky
\usepackage{tikz}
\usetikzlibrary{shapes,arrows}
% Složitější tabulky
\usepackage{tabularx}
\usepackage{multicol}

% Sázení osamocených float prostředí v horní části stránky
\makeatletter
\setlength{\@fptop}{0pt plus 10pt minus 0pt}
\makeatother

% Vynucení vypsání floating prostředí pomocí \FloatBarrier
\usepackage{placeins}

% Rámečky
\usepackage{mdframed}

%%%%%%%%%%%%%%
% MATEMATIKA %
%%%%%%%%%%%%%%

% Sázení matematiky a matematických symbolů ('\mathbb{}')
\usepackage{amsmath}
\usepackage{amssymb}
% Sázení fyzikálních veličin
\usepackage{siunitx}

%%%%%%%%%%%%%%%%%
% ZDROJOVÉ KÓDY %
%%%%%%%%%%%%%%%%%

% Sazba zdrojových kódů
\usepackage[formats]{listings}
% Přepnutí prostředí 'code' do režimu výpisu kódu
\newenvironment{code}{\captionsetup{type=listing}}{}

\lstset{
    basicstyle=\small\ttfamily,
    numbers=left,
    numberstyle=\tiny,
    tabsize=4,
    columns=fixed,
    showstringspaces=false,
    showtabs=false,
    keepspaces,
}

% Balíček 'minted' budeme používat pouze pokud je jeho hodnota nastavena na 'true'
\providecommand{\docminted}{false}
\ifthenelse{\equal{\docminted}{true}}
{
    % Sazba zdrojových kódů
    \usepackage[newfloat]{minted}
    % Nastavení barev 'minted' kódů
    \usemintedstyle{pastie}
}
{
    % \docminted není 'true', nic neprovádíme
    % Pokud je v dokumentu 'minted' prostředí, dokument se nepodaří přeložit.
}

%%%%%%%%%%%%%%%%%%%
% VLASTNÍ PŘÍKAZY %
%%%%%%%%%%%%%%%%%%%

\newcounter{todo}
\newcommand{\TODO}[1]{%
    \addtocounter{todo}{1}%
    \textcolor{red}{%
    \textbf{\sffamily\small{TODO \thetodo}%
    \ifthenelse{\equal{#1}{}}{}{:}%
    } %
    #1%
    }%
}

%%%%%%%%%%%
% TITULKA %
%%%%%%%%%%%

\newcommand{\titulka}{
    \vspace*{2em}
    \begin{center}
        \ifthenelse{\isundefined{\name}}{}{{\Huge \bfseries \name{}}}

        \ifthenelse{\isundefined{\subname}}{}{{\huge \bfseries \subname{}}}

        \vspace*{2em}

        \ifthenelse{\isundefined{\docdesc}}{}{{\Large \docdesc}}

        \vspace*{1em}

        \ifthenelse{\isundefined{\docgroup}}{}{\docgroup}

        \ifthenelse{\isundefined{\docurl}}{}{\url{\docurl}}
    \end{center}

    \vfill

    \ifthenelse{\isundefined{\authors}}{}{\authors{}}
    \ifthenelse{\isundefined{\corrections}}{}{\\\small (korektury \corrections{})}

    {}{\small \today}
    \\{\small FEKT.tex \fekttexversion{}}

    \thispagestyle{empty}
    \newpage
}

%%%%%%%%%%%%
% DOKUMENT %
%%%%%%%%%%%%

\begin{document}

\titulka{}

\tableofcontents
\footnote{Otázky zvýrazněné v seznamu otázek jsou napsány červeně}
\thispagestyle{empty}

\setcounter{page}{0}

\section{část 1}
\subsection{\textcolor{red}{Komunikační řetězec, vrstvový model datového přenosu, základní operace při zpracování signálu u digitálního komunikačního systému.}}
\subsubsection*{Komunikační řetězec}
\begin{itemize}
    \item Zdroj zprávy
    \item Kodovací zařízení/vysílač signálu
    \item  přenosová cesta
    \item  Dekodovací zařízení/přijímač signálu
    \item příjemce zprávy
\end{itemize}
\subsubsection*{Vrstvový model datového přenosu}
\textbf{RM-OSI}
\begin{itemize}
    \item \textbf{Aplikační vrstva} - Komunikace aplikačních procesů, funkce vrstvy může provádět uživatel, cílem je plnění požadavků uživatele
    \item \textbf{Prezentační vrstvy} - přizpůsobeních datových formátu pro prvky sítě, nezávisle na významu zprávy
    \item \textbf{Relační vrstva} - organizuje a synchoronizuje dialog mezi účastníky komunikace
    \item \textbf{Transportní vrstva} - řídí datové toky, segmentuje data
    \item \textbf{Síťová vrstva} - vytváří pakety, které směruje do určeného místa
    \item \textbf{Spojová vrstva} - Řídí komunikace po okruzích, odpovědná za zabezpečení přenosu dat
    \item \textbf{Fyzická vrstva} - vytváří a ruší fyzická spojení pro bitové toky
\end{itemize}

\subsubsection*{Základní operace při zpracování signálu u dig. komunikačního systému}

\begin{itemize}
    \item Format - digitalizace signálu
    \item Sourceen code- zdrojové kódování
    \item Encrypt
    \item Channel Encode
    \item Multiplex
    \item Pulse modulace - tvarování pulzů
    \item Bandpass modulace - modulace
    \item Frequency spread - kmitočtové rozprostření
    \item Multiplce access- mnohonásobný přístup
    \item XMT -vysílač
    \item Synchronizace
    \item RCV-přijímač
    \item Demodulace a vzorkování
    \item Detect - vyhodnocené symbolů
    
\end{itemize}


\subsection{\textcolor{red}{Úrovně signálu a vztažné hodnoty, absolutní a relativní úroveň, útlum, zisk, odstup signálu od šumu, výkonová spektrální hustota, přenosová kapacita kanálu.}}
\begin{itemize}
    \item \textbf{Absolutní úroveň} - srovnání veličiny ve sledovaném místě vzhledem k normálové hodnotě této veličiny (např. 1 mW)
    \item \textbf{Relativní} -srovnávání úrovně v určitém místě s úrovní ve vztažném místě
    \item \textbf{Útlum}- \textbf{A} - attenuation - \textbf{?} $S=-A$
    \item \textbf{Zisk}- G nebo S - gain $S=-A$
    \item \textbf{SNR}- signal to noise ration $SNR=\frac{S}{N}$
    \item \textbf{Výkonová spektrální hustota} - \textbf{PSD} - udává rozložení výkonu při přenosu signálu s náhodným charakterem 
    $PSD=\frac{P}{B}$
    \item \textbf{Přenosová kapacita kanálu} - C - množství informace, které lze přenést signálem o šířce pásma B, přijímaném výkonu S a šumu o výkonu N za jednotku času
    $$C=Blog_2(1+\frac{S}{N})$$
\end{itemize}

\subsection{Telekomunikační síť, struktura, způsoby komunikace, přenosové prostředky}
\begin{itemize}
    \item \textbf{Struktura} - určuje způsob komunikace, definuje protokoly a vrstvy
    \item \textbf{Způsoby komunikace}
    \item \textbf{Přenosové prostředky} \begin{itemize}
        \item Metalická vedení
        \item  Optická přenosová prostředí
        \item Radiové přenosová prostředí
    \end{itemize}
\end{itemize}


\subsection{\textcolor{red}{Metalická vedení, náhradní schéma homogenního vedení, primární parametry, sekundární parametry jednotky a vzájemné vztahy.}}
\begin{itemize}
    \item \textbf{Metalická vedení} - vedení tvořeno nejčasteji dvojicí souběžných metalických vodičů - dvě možné uspořádání:\begin{itemize}
        \item \textbf{Symetrické} - dvojice stočených vodičů v kabelu
        \item  \textbf{Koaxiální} - dvojice souosých vodičů
    \end{itemize} 
    \item Náhradní schéma homogenního vedení
    \item Promární paramtery
    \begin{itemize}
        \item \textbf{R} - odpor
        \item \textbf{L} - indukčnost
        \item  \textbf{G} - svod
        \item \textbf{C} - kapacita
        \item  pro dané vedení jsou konstatní
    \end{itemize}
    \item Sekundární parametry
    \begin{itemize}
        \item Nutné k sledování přenosových parametrů homogenního vedení
        \item \textbf{Měrný činitel přenosu} - $\gamma=\frac{\Delta U}{U*\Delta I}$
        \item \textbf{Charakteristická impedance} - $Z_c=\frac{U}{I}$
        \item  \TODO{jednotky}
    \end{itemize}
\end{itemize}


\subsection{\textcolor{red}{Konstrukce symetrických kabelových vedení používaných v přístupové síti, DM a x čtyřky.}}


\subsection{Modely elektrických parametrů kabelových vedení určené pro simulaci xDSL. (přehledově)}
\section{část 2}
\subsection{\textcolor{red}{Kryptografické metody zabezpečení datových přenosů, architektura bezpečnosti, služby bezpečnosti, mechanizmy bezpečnosti.}}
\subsection{Principy symetrických a asymetrických šifer, proudové a blokové šifry, hašovací funkce,\textcolor{red}{digitální podpis} .}
\section{část 3}
\subsection{\textcolor{red}{Princip zvyšování odolnosti přenášené zprávy proti chybám, informační poměr kódu, Hammingova vzdálenost, podmínky možnosti detekce a korekce chyb.}}
\subsection{\textcolor{red}{Schéma rozdělení protichybových kódů.}}
\subsection{\textcolor{red}{Schéma realizace procesu kódování blokových kódů.}}
\subsection{\textcolor{red}{Schéma realizace procesu kódování stromových kódů.}}
\subsection{Schéma kodéru cyklického blokového kódu, který je zadán vytvářecím mnohočlenem G(x).}
\subsection{RM kódy, jejich základní parametry.}
\subsection{Přehled možností používaných pro zabezpečení proti dlouhým shlukům chyb. (stručná charakteristika).}
\subsection{Obecné blokové schéma kodéru turbokódu, význam jeho částí}
\subsection{Dekódování Turbokódů.}
\subsection{\textcolor{red}{Vlastnosti ovlivňující návrh protichybového kódového systému. kap. 6. Kanálové kodovani 2.pdf}}
\subsection{Blokové schéma a časový diagram ARQ - SW.}
\subsection{Blokové schéma a časový diagram ARQ - návrat k bloku n (Go to n).}
\subsection{Blokové schéma a časový diagram ARQ s adresným opakováním.}
\subsection{Princip funkce a blokové schéma systému MRQ.}
\subsection{Postup při rozhodování o volbě PKS, tj. ARQ – FEC}
\subsection{Postup při rozhodování při smíšeném způsobu zabezpečení (HCS).}

\section{část 4}
\subsection{\textcolor{red}{DSL systémy, vlastnosti, referenční konfigurace, typické uspořádání přípojky, možnosti využití.}}
\subsection{Základní charakteristiky jednotlivých systémů xDSL, IDSL, HDSL, SDSL, ADSL, VDSL, vlastnosti, možnosti použití.}
\subsection{\textcolor{red}{DSL použité kódy a modulace, 2B1Q, QAM, TCM, DMT}}
\subsection{\textcolor{red}{ADSL principiální struktura modemu, význam jednotlivých bloků.}}
\subsection{\textcolor{red}{Metody zabezpečení proti chybám u ADSL.}}
\subsection{Přenosové protokoly využívané pro ADSL.}
\subsection{\textcolor{red}{Vliv rušení na provoz xDSL, kategorizace, dosažitelná přenosová rychlost, model přeslechů (NEXT, FEXT), princip výpočtu přeslechů.}}
\subsection{\textcolor{red}{Spektrální vlastnosti DSL přenosových systémů, správa spektra, cíle, metody.}}
\subsection{Metody potlačení ozvěn, funkce a princip vidlice.}
\subsection{\textcolor{red}{PLC systémy, princip, základní parametry, použité modulace, vazební členy, začlenění do sítě.}}
\end{document}
