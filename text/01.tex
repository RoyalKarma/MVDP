\section{část 1}
\subsection{\textcolor{red}{Komunikační řetězec, vrstvový model datového přenosu, základní operace při zpracování signálu u digitálního komunikačního systému.}}
\subsubsection*{Komunikační řetězec}
\begin{itemize}
    \item Zdroj zprávy
    \item Kodovací zařízení/vysílač signálu
    \item  přenosová cesta
    \item  Dekodovací zařízení/přijímač signálu
    \item příjemce zprávy
\end{itemize}
\subsubsection*{Vrstvový model datového přenosu}
\textbf{RM-OSI}
\begin{itemize}
    \item \textbf{Aplikační vrstva} - Komunikace aplikačních procesů, funkce vrstvy může provádět uživatel, cílem je plnění požadavků uživatele
    \item \textbf{Prezentační vrstvy} - přizpůsobeních datových formátu pro prvky sítě, nezávisle na významu zprávy
    \item \textbf{Relační vrstva} - organizuje a synchoronizuje dialog mezi účastníky komunikace
    \item \textbf{Transportní vrstva} - řídí datové toky, segmentuje data
    \item \textbf{Síťová vrstva} - vytváří pakety, které směruje do určeného místa
    \item \textbf{Spojová vrstva} - Řídí komunikace po okruzích, odpovědná za zabezpečení přenosu dat
    \item \textbf{Fyzická vrstva} - vytváří a ruší fyzická spojení pro bitové toky
\end{itemize}

\subsubsection*{Základní operace při zpracování signálu u dig. komunikačního systému}

\begin{itemize}
    \item Format - digitalizace signálu
    \item Sourceen code- zdrojové kódování
    \item Encrypt
    \item Channel Encode
    \item Multiplex
    \item Pulse modulace - tvarování pulzů
    \item Bandpass modulace - modulace
    \item Frequency spread - kmitočtové rozprostření
    \item Multiplce access- mnohonásobný přístup
    \item XMT -vysílač
    \item Synchronizace
    \item RCV-přijímač
    \item Demodulace a vzorkování
    \item Detect - vyhodnocené symbolů
    
\end{itemize}


\subsection{\textcolor{red}{Úrovně signálu a vztažné hodnoty, absolutní a relativní úroveň, útlum, zisk, odstup signálu od šumu, výkonová spektrální hustota, přenosová kapacita kanálu.}}
\begin{itemize}
    \item \textbf{Absolutní úroveň} - srovnání veličiny ve sledovaném místě vzhledem k normálové hodnotě této veličiny (např. 1 mW)
    \item \textbf{Relativní} -srovnávání úrovně v určitém místě s úrovní ve vztažném místě
    \item \textbf{Útlum}- \textbf{A} - attenuation - \textbf{?} $S=-A$
    \item \textbf{Zisk}- G nebo S - gain $S=-A$
    \item \textbf{SNR}- signal to noise ration $SNR=\frac{S}{N}$
    \item \textbf{Výkonová spektrální hustota} - \textbf{PSD} - udává rozložení výkonu při přenosu signálu s náhodným charakterem 
    $PSD=\frac{P}{B}$
    \item \textbf{Přenosová kapacita kanálu} - C - množství informace, které lze přenést signálem o šířce pásma B, přijímaném výkonu S a šumu o výkonu N za jednotku času
    $$C=Blog_2(1+\frac{S}{N})$$
\end{itemize}

\subsection{Telekomunikační síť, struktura, způsoby komunikace, přenosové prostředky}
\begin{itemize}
    \item \textbf{Struktura} - určuje způsob komunikace, definuje protokoly a vrstvy
    \item \textbf{Způsoby komunikace}
    \item \textbf{Přenosové prostředky} \begin{itemize}
        \item Metalická vedení
        \item  Optická přenosová prostředí
        \item Radiové přenosová prostředí
    \end{itemize}
\end{itemize}


\subsection{\textcolor{red}{Metalická vedení, náhradní schéma homogenního vedení, primární parametry, sekundární parametry jednotky a vzájemné vztahy.}}
\begin{itemize}
    \item \textbf{Metalická vedení} - vedení tvořeno nejčasteji dvojicí souběžných metalických vodičů - dvě možné uspořádání:\begin{itemize}
        \item \textbf{Symetrické} - dvojice stočených vodičů v kabelu
        \item  \textbf{Koaxiální} - dvojice souosých vodičů
    \end{itemize} 
    \item Náhradní schéma homogenního vedení
    \item Promární paramtery
    \begin{itemize}
        \item \textbf{R} - odpor
        \item \textbf{L} - indukčnost
        \item  \textbf{G} - svod
        \item \textbf{C} - kapacita
        \item  pro dané vedení jsou konstatní
    \end{itemize}
    \item Sekundární parametry
    \begin{itemize}
        \item Nutné k sledování přenosových parametrů homogenního vedení
        \item \textbf{Měrný činitel přenosu} - $\gamma=\frac{\Delta U}{U*\Delta I}$
        \item \textbf{Charakteristická impedance} - $Z_c=\frac{U}{I}$
        \item  \TODO{jednotky}
    \end{itemize}
\end{itemize}


\subsection{\textcolor{red}{Konstrukce symetrických kabelových vedení používaných v přístupové síti, DM a x čtyřky.}}


\subsection{Modely elektrických parametrů kabelových vedení určené pro simulaci xDSL. (přehledově)}