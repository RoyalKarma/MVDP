\section{část 3}
\subsection{\textcolor{red}{Princip zvyšování odolnosti přenášené zprávy proti chybám, informační poměr kódu, Hammingova vzdálenost, podmínky možnosti detekce a korekce chyb.}}
\subsection{\textcolor{red}{Schéma rozdělení protichybových kódů.}}
\subsection{\textcolor{red}{Schéma realizace procesu kódování blokových kódů.}}
\subsection{\textcolor{red}{Schéma realizace procesu kódování stromových kódů.}}
\subsection{Schéma kodéru cyklického blokového kódu, který je zadán vytvářecím mnohočlenem G(x).}
\subsection{RM kódy, jejich základní parametry.}
\subsection{Přehled možností používaných pro zabezpečení proti dlouhým shlukům chyb. (stručná charakteristika).}
\subsection{Obecné blokové schéma kodéru turbokódu, význam jeho částí}
\subsection{Dekódování Turbokódů.}
\subsection{\textcolor{red}{Vlastnosti ovlivňující návrh protichybového kódového systému. kap. 6. Kanálové kodovani 2.pdf}}
\subsection{Blokové schéma a časový diagram ARQ - SW.}
\subsection{Blokové schéma a časový diagram ARQ - návrat k bloku n (Go to n).}
\subsection{Blokové schéma a časový diagram ARQ s adresným opakováním.}
\subsection{Princip funkce a blokové schéma systému MRQ.}
\subsection{Postup při rozhodování o volbě PKS, tj. ARQ – FEC}
\subsection{Postup při rozhodování při smíšeném způsobu zabezpečení (HCS).}
