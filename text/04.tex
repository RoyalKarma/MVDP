\section{část 4}
\subsection{\textcolor{red}{DSL systémy, vlastnosti, referenční konfigurace, typické uspořádání přípojky, možnosti využití.}}
\subsection{Základní charakteristiky jednotlivých systémů xDSL, IDSL, HDSL, SDSL, ADSL, VDSL, vlastnosti, možnosti použití.}
\subsection{\textcolor{red}{DSL použité kódy a modulace, 2B1Q, QAM, TCM, DMT}}
\subsection{\textcolor{red}{ADSL principiální struktura modemu, význam jednotlivých bloků.}}
\subsection{\textcolor{red}{Metody zabezpečení proti chybám u ADSL.}}
\subsection{Přenosové protokoly využívané pro ADSL.}
\subsection{\textcolor{red}{Vliv rušení na provoz xDSL, kategorizace, dosažitelná přenosová rychlost, model přeslechů (NEXT, FEXT), princip výpočtu přeslechů.}}
\subsection{\textcolor{red}{Spektrální vlastnosti DSL přenosových systémů, správa spektra, cíle, metody.}}
\subsection{Metody potlačení ozvěn, funkce a princip vidlice.}
\subsection{\textcolor{red}{PLC systémy, princip, základní parametry, použité modulace, vazební členy, začlenění do sítě.}}